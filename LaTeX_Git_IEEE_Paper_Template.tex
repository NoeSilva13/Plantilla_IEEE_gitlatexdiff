\documentclass[journal]{IEEEtran}
\usepackage{xcolor,soul,framed} %,caption
\colorlet{shadecolor}{yellow}
\usepackage[pdftex]{graphicx}
%\graphicspath{{../pdf/}{../jpeg/}}
%\DeclareGraphicsExtensions{.pdf,.jpeg,.png}
\usepackage[cmex10]{amsmath}
\usepackage{array}
\usepackage{mdwmath}
\usepackage{mdwtab}
\usepackage{eqparbox}
\usepackage{url}
\hyphenation{op-tical net-works semi-conduc-tor}

%Title and authors
\begin{document}
\bstctlcite{IEEEexample:BSTcontrol}
    \title{An amazing title for your manuscript}
  \author{First Author$^{1,2}$
      Second Author$^{2}$\\
      Third Author$^{1}$
      and Corresponding Author$^{1}$,~\IEEEmembership{Member,~IEEE,}

$^1$ Instituto Polit\'ecnico Nacional, UPALM, Edif. Z-4 3er Piso, Cd. de M\'exico, 07738, M\'exico \\
$^2$Departament d’Enginyeria Electrònica, Universitat Autònoma de Barcelona, Bellaterra, 08193, Spain \\
email: authorname@ipn.mx
}
\maketitle

% ======================================= ABSTRACT =========================================
% ==========================================================================================
\begin{abstract}
%\boldmath
This manuscript ...

\end{abstract}

% ======================================= KEYWORDS =========================================
% ==========================================================================================
\begin{IEEEkeywords}
Keyword 1, Keyword2, ...
\end{IEEEkeywords}

\section{Introduction}

\IEEEPARstart{I}{n} the next years...  \\


\section{Title of Section II}

The proposed approach ...

\section{Title of Section III}

Some text ...

\section{Title of Section IV}

Use as many sections as necessary.

\section{Conclusion}

What we have learn?



\section*{Acknowledgment}



The authors would like to thank ...

\begin{thebibliography}{1}
    \bibliographystyle{IEEEtran}

    \bibitem{Reference1} Reference 1
    
\end{thebibliography}

\end{document}